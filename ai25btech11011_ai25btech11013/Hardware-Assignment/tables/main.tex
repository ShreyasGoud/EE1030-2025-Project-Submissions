\let\negmedspace\undefined
\let\negthickspace\undefined
\documentclass[journal]{IEEEtran}


\setlength{\headheight}{1cm} % Set the height of the header box
\setlength{\headsep}{0mm}     % Set the distance between the header box and the top of the text
 \usepackage[a4paper,margin=10mm, onecolumn]{geometry}
\usepackage{gvv-book}
\usepackage{gvv}
\usepackage{cite}
\usepackage{amsmath,amssymb,amsfonts,amsthm}
\usepackage{algorithmic}
\usepackage{graphicx}
\usepackage{textcomp}
\usepackage{xcolor}
\usepackage{txfonts}
\usepackage{listings}
\usepackage{enumitem}
\usepackage{mathtools}
\usepackage{gensymb}
\usepackage{comment}
\usepackage[breaklinks=true]{hyperref}
\usepackage{tkz-euclide} 
\usepackage{listings}                                       
\def\inputGnumericTable{}                                
\usepackage[latin1]{inputenc}                                
\usepackage{color}                                            
\usepackage{array}                                            
\usepackage{longtable}                                       
\usepackage{calc}                                             
\usepackage{multirow}                                         
\usepackage{hhline}                                           
\usepackage{ifthen}                                           
\usepackage{lscape}
\usepackage{circuitikz}
\tikzstyle{block} = [rectangle, draw, fill=blue!20, 
    text width=4em, text centered, rounded corners, minimum height=3em]
\tikzstyle{sum} = [draw, fill=blue!10, circle, minimum size=1cm, node distance=1.5cm]
\tikzstyle{input} = [coordinate]
\tikzstyle{output} = [coordinate]

\begin{document}

\bibliographystyle{IEEEtran}
\vspace{3cm}

\title{Hardware-Assignment Table}
\author{AI25BTECH11011-VARUN \\
AI25BTECH11013-GAUTHAM}
 \maketitle
% \newpage
% \bigskip
{\let\newpage\relax\maketitle}

\renewcommand{\thefigure}{\theenumi}
\renewcommand{\thetable}{\theenumi}
\setlength{\intextsep}{10pt} % Space between text and floats


\numberwithin{equation}{enumi}
\numberwithin{figure}{enumi}
\renewcommand{\thetable}{\theenumi}
\begin{table}[h!]
\centering
\caption{Training Data (Voltage vs Temperature)}
\begin{tabular}{|c|c|}
\hline
	\textbf{Voltage (V)} & \textbf{Temperature (\degree{C})} \\ 
\hline
2.6344 & 92.0 \\
2.6540 & 86.2 \\
2.6637 & 83.1 \\
2.6735 & 79.9 \\
2.6931 & 74.1 \\
2.7175 & 71.2 \\
2.7273 & 67.8 \\
2.7468 & 62.6 \\
2.7664 & 57.5 \\
2.7859 & 52.3 \\
2.8104 & 47.0 \\
2.8299 & 41.8 \\
2.8543 & 37.6 \\
2.8739 & 34.8 \\
2.9032 & 31.5 \\
\hline
\end{tabular}
	\caption{Training Data Temperature and Voltage}
	\label{tab:Training Data}
\end{table}

%------------------------%
% Validation Data Table
%------------------------%
\begin{table}[h!]
\centering
\caption{Validation Data (Measured vs Predicted Temperature)}
\begin{tabular}{|c|c|c|}
\hline
	\textbf{Voltage (V)} & \textbf{Measured Temp (\degree{C})} & \textbf{Predicted Temp (\degree{C})} \\ 
\hline
2.6442 & 88.6 & 89.06 \\
2.6637 & 81.3 & 83.48 \\
2.7028 & 72.9 & 72.84 \\
2.7370 & 65.1 & 64.14 \\
2.7566 & 60.0 & 59.41 \\
2.7761 & 54.6 & 54.90 \\
2.8006 & 49.0 & 49.48 \\
2.8201 & 44.3 & 45.38 \\
2.8495 & 38.8 & 39.55 \\
2.8837 & 32.8 & 33.30 \\
\hline
\end{tabular}
        \caption{Validation Data Predicted Temperature, Actual Temperature and Voltage}
	\label{tab:Validation Data}
\end{table}

\end{document}

